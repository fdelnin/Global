\documentclass{article}
\usepackage[utf8]{inputenc}
\usepackage{amsmath}
\usepackage{amssymb}
\usepackage{mathtools}
\usepackage{latexsym}
\usepackage{verbatim}
\usepackage{amsthm}
\usepackage{amsmath}
\newcommand{\pl}{P_{\backslash L}}
\newcommand{\pil}{P'_{\backslash L}}
\newcommand{\passo}{\xrightarrow{\alpha}}
\newcommand{\passof}{\xrightarrow{f(\alpha)}}
\newcommand{\pd}{P_{2}}

\title{Esercizi esame Linguaggi per il Global Computing}
\author{Francesca Del Nin}
\date{June 2019}

\begin{document}

\maketitle

\section{Esercizio B}
\subsection{Testo}

Dimostrare che ogni processo del CCS finito termina in un numero finito di passi.
\begin{center}
    $P, Q = ~0 ~|~ \alpha .P~|~ P+Q~|~ P|Q ~ | ~P_{\backslash L}~|~P[f] $
\end{center}

\subsection{Dimostrazione con somme finite}
Voglio dimostrare che  $\forall S \in CCS~  se ~ S \xrightarrow{\alpha}  S', ~S'~  finito ~ \implies  ~ S ~ termina~  in~  un~  numero~  finito ~ di~  passi$, definisco $size(S)$ come il numero di passi del processo S:

\begin{align*}
size(0) &= 0 \\
size(\alpha.P) &= 1+size(P)\\
size(P|Q) &= size(P)+size(Q)\\
size(P+Q) &= max\{size(P), size(Q)\}\\
size(P_{\backslash L}) &= size(P)\\
size(P[f]) &= size(P)
\end{align*}

E' anche possibile dimostrare che il numero di stati raggiungibili da S è finito, per fare questo definisco B come il bound superiore al numero di stati raggiungibili da P:

\begin{align*}
B(0) &= 1\\
B(\alpha.P) &= 1 + B(P)\\
B(P|Q) &= B(P) * B(Q)\\
B(P+Q) &= B(P) + B(Q)\\
B(\pl) &=  B(P)\\
B(p[f]) &=  B(P)
\end{align*}

\paragraph{Dimostrazione} %$    \forall S \in CCS~  se ~ S~  finito ~ \Rightarrow ~ S ~ termina~  in~  un~  numero~  finito ~ di~  passi$ ovvero: $ S  ~finito,~ S \xrightarrow{\alpha}  S' ~ \Rightarrow~ size(S) > size(S') $

Dimostro che se $ S \xrightarrow{\alpha}  S' ~ \implies ~ size(S) > size(S') $ e che se i sottoprocessi di S sono finiti (size S' è finita) allora S termina.
Dimostrazione per induzione sulla struttura di S, l'ipotesi e che se i sottoprocessi di S terminano in un numero finito di passi allora anche S termina in un numero finito di passi.
Inoltre è possibile dimostrare che anche il numero di stati raggiungibili è finito, l'ipotesi è che se il numero di stati stati raggiungibili dai sottoprocessi di S è finito, allora anche gli stati raggiungibili da S sono finiti.

\paragraph{Caso Base S = 0}: 

S non esegue nessun passo per cui $S \not\passo$  $\implies$ S termina in un numero  finito di passi. Inoltre $B(0) = 1$ e $size(0) = 0$

\paragraph{Caso Induttivo Prefix S = $\alpha$.P}:

Per la regola ACT $\alpha.P \xrightarrow{\alpha}P$ e per ipotesi induttiva so che P termina in un numero finito di passi (size(P) è finita). $size(S) = 1+ size(P)$ quindi anch'essa finita perche si aggiunge un passo ai passi (finiti) di P.

Inoltre è possibile dimostrare che anche il bound al numero di stati raggiungibili è finito dato che $B(P)$ è finito per ipotesi induttiva è $B(S) = B(P)+1$ e quidi finito.

\paragraph{Caso Induttivo Parallelo S = P$|$Q}:

$size(S) = size(P) + size(Q) $ e $B(S) = B(P) * B(Q)$. Ci sono tre casi in basse alle regole PAR:
\begin{itemize}
    \item[$PAR_{\backslash L}$] $S = P|Q \xrightarrow{\alpha} P'|Q$ e $S'=P'|Q$, so per ipotesi induttiva che $P$ e $Q$ sono finiti e terminano in un numero finito di passi, e so che $size(S') = size(P') + size(Q) $ e so che la premessa alla regola SUM1  $P\xrightarrow{\alpha} P'$ vale. Quindi $size(P) = 1 + size(P')$ (perche fa un passo da P a P') e si ha che $size(S) = size(P) + size(Q) = 1+size(P') + size(Q) > size(P') + size(Q) = size(S')$. 
    
    Per ipotesi induttiva $P$ e $Q$ terminano per cui anche $S$ termina.
    
    Inoltre $B(S) = B(P)*B(Q)$  e per ipotesi B(P) e B(S) sono finiti per cui anche il limite superiore al numero di stati raggiungibili da S è finito.
    
     \item[$PAR_{\backslash R}$] Analogo con P e Q scambiati
     
     \item[$SINC$] $S = P|Q \xrightarrow{\alpha} P'|Q'$ e $S'=P'|Q'$, so per ipotesi induttiva che P' e Q' sono finiti. In questo caso i due processi si sincronizzano eseguendo un passo, quindi $size(S) = 1 + size(S') > size(S')$ e S termina in un numero finito di passi. 
     
     Inoltre $B(S) = B(P) * B(Q)$ e per ipotesi $B(P)$ e $B(Q)$ sono finiti per cui anche $B(S)$ lo è.
\end{itemize}

\paragraph{Caso Induttivo Non Determinismo S = P$+$Q}:

$size(S) = max\{size(P), size(Q)\}$ e $B(S) = B(P) + B(Q)$, ci sono due casi in base alle regole SUM:

\begin{itemize}
    \item [$SUM_{\backslash L}$] in questo caso $S = P+Q \xrightarrow{\alpha} P' = S'$, ovvero S fa un passo e va in P', in questo caso si ha $size(S) = max\{size(P), size(Q)\}$ e a sua volta ci sono due casi:
    \begin{itemize}
        \item [Max=P] in questo caso $size(S) = size(P)$ e $S \passo S'=P'$ quindi $size(S')=size(P')$ e siccome P ha fatto un passo per arrivare in P' è vero anche che $size(P) = 1 + size(P')$ e quindi $size(S)=size(P)>size(P')=size(S')$. 
        
        Inoltre per ipotesi induttiva $Q$ e $P$ terminano per cui anche $S$ termina.
        
        \item[Max=Q] in questo caso $size(S) = size(Q)$ e  $S \passo S'=P'$ quindi $size(S')=size(P')$. Sappiamo che $size(Q)>size(P)$ e che P fa un passo per andare in P'. Si deduce che $size(S)=size(Q)>size(P)>size(P')=size(S')$.
        Inoltre per ipotesi induttiva $Q$ e $P$ terminano per cui anche $S$ termina.
        
    \end{itemize}
    
    \item [$SUM_{\backslash R} $] analogo con P e Q scambiati.
    
In tutti i casi si ha che $ B(S) =B(P) + B(Q) $ e per ipotesi sappiamo che $ B(P) $ e $ B(Q) $ sono finiti, per cui anche $ B(S) $ è finito.
\end{itemize}

\paragraph{Caso Induttivo Restriction S=$\pl$}:

In questo caso $size(S) = size(P)$ e $B(S)=B(P)$. 

Per la regola RES $\pl \passo \pil$ quindi $size(S')= size(\pil) =size(P')$. Sappiamo che P fa un passo ($\alpha \not\in L$) e va in P', per cui $size(P) = 1+ size(P')>size(P') \implies size(S)>size(S')$. Per ipotesi induttiva $size(P')$ è finita per cui S termina.

Inoltre $B(S) =B(P)$ e $B(S') =B(P')$ per ipotesi si ha che $B(P')$ è finito (in quanto sottoprocesso di S) per cui anche $B(S)$ lo è.

\paragraph{Caso Induttivo Relabeling S=P[f]}:


Anche in questo caso $ size(S) = size(P) $ e $ B(S)=B(P) $. 

Per la regola REL $P[f] \passof P'[f] =S'$, e $size(S') = size(P') $. Per la premessa alla regola sappiamo che P fa un passo e va in P', per cui si ha che $size(P) = 1+size(P')$ quindi $size(S)>size(S')$, inoltre per ipotesi induttiva $P'$ termina in un numero finito di passi per cui anche $S$ termina.

Anche il numero di stati raggiungibili è finito perché per ipotesi $B(P')$ è finito in quanto sottoprocesso di P(=S) e $B(S)=B(P)=B(P')$ per cui è finito.



\subsection{Dimostrazione con somme infinite}

In questo caso non è possibile dimostrare che il numero di stati raggiungibili è finito perché il bound superiore al numero di stati è definito come la somma tra tutti i bound dei processi sommati, ovvero $B(P+Q)=B(P)+B(Q)$, quindi nel caso di una scelta tra un numero infinito di processi la somma sarebbe infinita.

Si può comunque dimostrare che i processi che contengono scelte non deterministiche tra un numero infinito di processi terminano.\\

Per fare ciò utilizzo la struttura della grammatica che genera CCS,% è possibile generare somme infinite utilizzando la grammatica sopra indicata perché $$\sum_{i}^{3} Pi = \sum_{i}^{2} Pi + P_{3}$$ voglio quindi dimostrare che per ogni lunghezza della derivazione, ovvero del processo generato dalla grammatica, il processo termina in un numero finito di passi.
l'ipotesi induttiva è quindi che se il processo generato ha lunghezza finita (indipendentemente dalla presenza di somme infinite) allora termina in un numero finito di passi.

Procedo per induzione sula lunghezza della derivazione:

\subsubsection{Caso Base}
In questo caso la lunghezza è 0, l'unico processo di lunghezza 0 è $0$, che non ha passi possibili e $size(0)=0$.

\subsubsection{Casi Indittuvi}
in questo caso la lunghezza della dervazione è $n+1$ e per ipotesi sappiamo che fino alla lunghezza $n$ è vero che il processo termina in un numero finito di passi, ovvero $size($derivaizone fino ad $n) = k$ con $k$ finito.

\paragraph{Caso $S=\alpha .P$}
In questo caso alla derivazione di $P$ stiamo "aggiungendo" $\alpha$ e quindi $size(S)=1+size(P)$ e siccome $size(P)$ è finita per ipotesi induttiva anche $size(S)$ lo è.

\paragraph{Caso $S= P|Q$}
$P$ e $Q$ sono processi che terminano in un numero finito di passi, ovvero la loro funzione $size$ è finita. Sappiamo che $size(P+Q)=size(P)+size(Q)$ quindi è finita.

\paragraph{Caso $S=\pl$}
In questo caso sappiamo che $P$ è finito e applichiamo una restrizione a P, il processo $S$ termina quindi in un numero finito di passi perchè $P$ termina per ipotesi e $size(S)=size(P)$.

\paragraph{Caso $S=S[f]$}
$P$ termina per ipotesi induttiva e stiamo applicando una funzione di relabeling ai canali di $P$, $size(S)=size(P)$ e quindi termina in un numero finito di passi dato che $P$ termina in un numero finito di passi per ipotesi.

\paragraph{Caso $S= \sum_{i \in I}P_{i}$}
in questo caso la somma è infinta, ci sono quindi infiniti $P_{i}$ che vengono sommati, per ipotesi induttiva ognuno dei $P_{i}$ termina in un numero finito passi ovvero $size(P_{i})=k$ con $k$ finito. $size(S) = max\{size(P_{i})| i \in I\}$, ma sappiamo che ogni $size(P_{i})$ è finita per cui anche $size(S)$ è finita.\\\\\\

Abbiamo quindi dimostrato che sia nel caso finito che nel caso infinito i processi CCS terminano in un numero finito di passi.



\begin{comment}

%Per fare ciò utilizzo la struttura della grammatica, %è possibile generare somme infinite utilizzando la grammatica sopra indicata perché $$\sum_{i}^{3} Pi = \sum_{i}^{2} Pi + P_{3}$$  quindi
Dimostro che se un processo generato dalla grammatica termina in un numero finito di passi, allora aggiungendo un termine il processo termina comunque in un numero finito di passi.

L'ipotesi induttiva è che se un processo $P$ (e $Q$) termina con un numero finito di passi, e $P$ (e $Q$) sono sottoprocessi di $S$,  allora anche $S$ termina in un numero finito di passi. Ovvero se P ha derivazione finita allora termina, l'ipotesi è che Se P termina anche S termina.

\paragraph{Caso Base} Il caso base è il termine $0$, in questo caso il processo termina in un numero finito di passi.

\paragraph{Caso Induttivo S=$\alpha$.P} In questo caso si aggiunge un passo $\alpha$ al processo $P$ che è termina per ipotesi, per cui anche $S$ termina.

\paragraph{Caso Induttivo S=Q$|$P}
In questo caso $Q$ e $S$ sono processi che terminano per ipotesi, per cui S termina dato che il limite superiore al numero di passi è $size(S) = size(P)+size(Q)$.


\paragraph{Caso Induttivo S=P[f]} In questo caso si applica la funzione di relabeling al processo $P$ che per ipotesi termina, quindi anche $S$ termina dato che size(s)=$size(P)$

\paragraph{Caso Induttivo S=\pl} In questo caso si applica la funzione di restriction al processo $P$ che per ipotesi termina, in questo caso anche S termina dato che $size(S)=size(P)$


\paragraph{Caso Induttivo S=$\sum_{i\inI}Pi$} In questo caso $S=P_{1}+P_{2}+P_{3}+...$ e per ipotesi induttiva sappiamo che ogni $Pi$ è finito e termina in un numero finito di passi. In questo caso il processo $S$ fa un passo e va in $S'=Pi$ per qualche $i \in I$, quindi $size(S') = size(Pi)$ che per ipotesi è finita dato che il processo termina; $size(S)=max\{size(Pi), i\in I\}$ ma siccome ogni processo $Pi$ è finito per ipotesi, il massimo elemento è comunque finito, e quindi il processo termina in un numero finito di passi dato se si aggiunge un passo (quello della scelta).


Alt: se ci sono $0$ processi il caso base è banale. Per $n+1$ processi sappiamo che per $n$ è vero per ipotesi induttiva$_{2}$ e aggiungendo un processo $P_{n+1}$, quindi la scelta si riduce tra gli $n$ programmi precedenti e l'$n+1$esimo, ma $P_{n+1}$ ha derivazione di lunghezza finita per ipotesi, per cui siccome il numero di passi di S è delimitato dal numero di passi del più grade processo $P_{i}$ all'interno della somma, se tutti sono finiti anche $S$ sarà finito.

%Dimostro che S termina per induzione sul numero di processi nella somma:
%\begin{itemize}
%    \item [Caso Base] se sommo 0 processi il processo termina.
 %   \item [Casi induttivi] assumo che $S=\sum_{n}Pi$ termini in un numero finito di passi, se aggiungo il processo Q alla scelta, ovvero $S'= S+Q = \sum_{n}Pi+Q$ il processo S 
%\end{itemize}
\end{comment}




\section{Esercizio Q}
\subsection{Testo}
Dimostrare il teorema di Knaster-Tarski nel caso di reticoli completi


\paragraph{Enunciato Teorema}

Sia L un reticolo completo e $f:L \to L$ una funzione monotona $\Rightarrow$ l'insieme dei punti fissi di $f$ in $L$ è un reticolo completo.

\paragraph{Def. Upper Bound}: sia $S$ un insieme di numeri reali, $x$ è un upper bound di $S$ se $x \geq s \;\forall s \in S$ 

\paragraph{Def. Least Upper Bound}: $x$ è un Least Upper Bound di $S$ se $x \leq y \; \forall y $ upper bounds di $S$

\paragraph{Def. Insieme Parzialmente Ordinato}: è un insieme $I$ su cui per ogni sua coppia di elementi vale una relazione binaria $\sqsubseteq$ che soddisfa i seguenti assiomi:
\begin{itemize}
\item riflessività (ogni elemento è in relazione con sè stesso)
\item antisimmetria ($a \leq b, \; b \leq a \implies a=b$)
\item transitività ($a \leq b, \; b \leq c \implies a \leq c$)
\end{itemize}



\paragraph{Def. Reticoli Completi}: un insieme parzialmente ordinato L in cui ogni sottoinsieme ha $lub$ e $glb$ in L.

%TODO add def funzione monotona
\subsection{Dimostrazione}
Per dimostrare che l'insieme dei punti fissi di $f$ in $L$ è un reticolo completo dimostro che dato $\langle P,\sqsubseteq\rangle$ l'insieme dei punti fissi di $f$:
\begin{itemize}
    \item sia $P_1 =\{x \in L | x  \sqsubseteq f(x)\}$, il $lub$ di $P_1$ = $gfp$ (greatest fixed point) di $f$ e quindi $\in P$
    \item sia $P_2 =\{x \in L | f(x)  \sqsubseteq x\}$, il $glb$ di $P_2$ = $lfp$ (least fixed point) di $f$ quindi $\in P$
    \item ogni sottoinsieme dei punti fissi di $f$ in $L$ ha $lub$ e $glb$ ed essi $\in P$
\end{itemize}

I primi due punti dimostrano il Lemma di Knaster Tarski e il terzo dimostra che è un complete lattice.

\paragraph{Punto 1}
%\subsection{Knaster Tarski in CCS}

Dimostro che dato $P_{1}=\{x \in L | x  \sqsubseteq f(x)\}$ ovvero l'insieme di tutti i postfix points, il $lub$ di tale insieme è il greatest fixed point di $f$ (ed appartiene a $P1$).

%Si ha che $\forall x \in P_{1}$ vale $x \sqsubseteq f(x)$ per definizione, per monotonia di $f$ si ha che $f(x)\sqsubseteq f(f(x))$ $\implies$ $f(x) \in P_{1}$ 

Sia $u=\sqcup P_{1}$ allora $\forall x \in P_{1} \; x \sqsubseteq u$ e per monotonia di $f$ vale anche che $\forall x \in P_{1} \; f(x)\sqsubseteq f(u)$ e per definizione di $P_{1}\; x \sqsubseteq f(x)$, questo indica che $f(u)$ è un upper bound dell'insieme $P_{1}$ (perché $x \sqsubseteq f(x) \sqsubseteq f(u)$ ). Ma $u$ è il least upper bound quindi $$u \sqsubseteq f(u) \implies u \in P_{1}$$.

Dato che $u \in P_{1}$ per monotonia di $f$ si ha che $$f(u)\sqsubseteq f(f(u)) \implies f(u) \in P_{1}$$
 quindi $f(u)\sqsubseteq u$ Allora:
\begin{itemize}
    \item $u\sqsubseteq f(u)$
    \item $f(u)\sqsubseteq u$
\end{itemize}
$\implies u=f(u)=\sqcup P_{1}$ e $\in P_{1}$.

Siccome ogni fixed point appartiene all'insieme $P_{1}$, perché $$\forall n \in P \; f(n)=n  \;\text{ovvero} \; n \sqsubseteq f(n) \implies n \in P_{1}$$ e $u=f(u)=\sqcup P_{1}$ è il $lub$ di tale insieme, allora $\forall n \in P \; n \in P_{1} $ e $$n = f(n) \sqsubseteq u = f(u) =\sqcup P_{1}$$

Ovvero $u=f(u)$  è il più grande fixed point di $f$ ed è il $least\; upper\; bound$ di $P_{1}$ ed $\in P$ in quanto è un fixed point. 

\paragraph{Punto 2} Dimostro che dato $\pd = \{x \in L | x \sqsupseteq f(x) \}$ ovvero l'insieme dei prefixed point di $f$ in $L$, il $glb$ di tale insieme è uguale al least fixed point di $f$.

%E' vero che $\forall x \in \pd \; x \sqsupseteq f(x)$ e per monotonia di $f$: $f(x) \sqsupseteq f(f(x)) \implies f(x) \in \pd$.

Sia $l = \sqcap \pd $ allora $l \sqsubseteq x \; \forall x \in \pd$. Per monotonia di $f$ è vero che $\forall x \in \pd \; f(l) \sqsubseteq f(x)$ ovvero $f(l)$ è un lower bound di $\pd$.

$l$ è il greatest lower bound di $\pd$ quindi $l \sqsupseteq f(l) $, allora $l \in \pd$ ($l$ è in L perché $f:L\to L$).

Per monotonia di $f$: $f(l) \sqsupseteq f(f(l)) \Rightarrow f(l) \in \pd$ e quindi $l\sqsubseteq f(l)$ in quanto $l$ è greatest lower bound di $\pd$, ovvero è minore o uguale a tutti gli elementi di $\pd$. Quindi:
\begin{itemize}
    \item $l\sqsupseteq f(l)$
    \item $f(l) \sqsupseteq l$
\end{itemize}
$\implies l = f(l) = \sqcap \pd $. $ l $ è chiaramente un fixed point, e siccome tutti i fixed point appartengono a $ \pd $ perchè $ \forall n \in P \; f(n) = n $ è vero anche che $ n \sqsupseteq f(n) \implies n \in P_{2} $ e $l=f(l)$ è il greatest lower bound di $\pd$ $\implies f(l) \sqsubseteq x \; \forall x \in \pd $ quindi è più piccolo di tutti gli elementi in $\pd$ quindi anche di tutti i fixed point, ovvero è il least fixed point di $f$ in $L$ ed appartiene a $L$ perché $f$ ha dominio in $L$.

\paragraph{Punto 3}
Devo dimostrare che per ogni sottoinsieme di $P$ esistono $lub$ e $glb$ e cha appartengono a $P$, ovvero che sono punti fissi.

Sia $ S \subseteq P$ e $ q = \sqcup S$ (least upper bound di S), $q$ esiste perché $S \subseteq P \subseteq L$ e L è un complete lattice, quindi per definizione di complete lattice ogni suo sottoinsieme ha lub e glb in $L$.\\\\
Sia inoltre $$I=\{x \in L | q \sqsubseteq x\}$$

\subparagraph{3.1}Si ha che (per riflessività della relazione d'ordine) $q \sqsubseteq q \implies q \in I$.

($I$ è sottoinsieme di $L$, quindi ha un glb, per assurdo supponiamo che esista un $q'$ streattamente più grande di $q$ e che sia il glb, ma allora se è streattamente maggiore di $q$ $q'\in I$ ma quindi $q$, che appartiene a $I$ è più piccolo $\implies q'$ NON è il glb  )

Per defizione $q$ è il glb dell'insieme $I$, in quanto è il più grande elemento che è più piccolo di ogni elemento appartente all'insieme, $$q=glb(I)= \sqcap I \implies glb(I) \in I$$

\subparagraph{3.2}

$I$ è un sottoinsieme e $I\subseteq L$ quindi ha un lub in $L$, inoltre è vero che $$\forall x \in I \; x \sqsubseteq lub(I)$$ 

ma sappiamo anche che $$\forall x \in I \; q \sqsubseteq x$$ 
Allora
$$\implies q \sqsubseteq x \sqsubseteq lub(I) \implies lub(I) \in I$$

\subparagraph{3.3}

$I$ è un intervallo perchè contiene tutti gli elementi da $q$ a $Top $ element  di $L$, inoltre $I$ è sottoinsieme di $L$ complete lattice, quindi $I$ è un complete lattice.

Ora dimostro che $f$ definita su $I$ ha codominio $I$, ovvero $f$ è chiusa rispetto a $I$, (se questo è vero posso applicare i punti 1 e 2 ecc). 

Per definizione di $I$ e di $S$ si ha che $$ \forall x \in I, \; \forall s \in S \; s \sqsubseteq q\; \text{e}\; q \sqsubseteq x  $$

Per monotonia di $f$ vale quindi $f(s)\sqsubseteq f(q)\sqsubseteq f(x)\; \forall x \in I, \; \forall s \in S  $
$$\implies f(s) \sqsubseteq f(x)$$
Per definizione di $S$ ogni elemento in $S$ è un punto fisso (sottoinsieme di $P$), quindi $$\forall s \in S \; \forall x \in I \; f(s)=s\sqsubseteq f(x)$$

Quindi $f(x)$ è un upper bound dell'insieme $S$, $\implies q=lub(S) \sqsubseteq f(x) $ ovvero $q\sqsubseteq f(x) \forall x \in I$. ovver $$ \forall x \in I \; q \sqsubseteq f(x) \implies f(x) \in I$$

Ora so che: $I$ è un complete lattice in quanto intervallo in un complete lattice, $\forall x \in I \; f(x):I \to I$, posso quindi applicare il punto \textbf{2} e vale che 
\begin{center}
$lfp$ di $f$ in $I \in I$ (e $P$)
\end{center}

Allora $q= \sqcup S \in P$. 

\begin{comment} 1 Sia $ S \subseteq P$ e $ s = \sqcup S$ (least upper bound di S), dimostro che esiste un elemento di P che è maggiore di tutti gli elementi di S, e che tale elemento è il più piccolo elemento di P (comunque più grande di S).

Definisco $U = \{ x\in P| s \sqsubseteq x\}$ ovvero l'insieme degli elementi più grandi di $s$ in $P$ (upper closure). $U$ è quindi l'intervallo degli elementi da $s$ al Top element di $L$: $U=[s....\top]$. Si ha che $s \in U$.

Ora dimostro che U è chiuso rispetto a $f$, ovvero:
\paragraph{Definizione di chiusura} sia $f: S \to T$, e $S'\subseteq S$, allora si dice che $S'$ è chiuso rispetto ad $f$ se e solo se:
$f[S'] \subseteq S'$, dove $f[S']$ è l'immagine di $f$ in $S$.\\

Quindi voglio dimostrare che $f[U] \subseteq U$. Dimostro che: 
\begin{itemize}
    \item[\textbf{a}] $s \sqsubseteq f(s)$
    \item[\textbf{b}] $\forall x \in U$ $s \sqsubseteq f(x)$
\end{itemize}

\paragraph{a} $\forall e \in S$ vale che $ e \sqsubseteq s$ per definizione di lub. Dato che $S$ è un sottoinsieme dei punti fissi vale che $ e = f(e)$ e per monotonia di $f $ vale che $f(e) \sqsubseteq f(s) \implies \forall e \in S. e \sqsubseteq f(s)$, ovvero $f(s)$ è un upper bound di S, ma per definizione di least upper bound si ha che $s \sqsubseteq f(s)$.

\paragraph{b} $\forall x \in  U$ $s \sqsubseteq x$ per definizione di $U$. Per monotonia di $f$: $f(s) \sqsubseteq f(x)$, per \textbf{a} $s \sqsubseteq f(s) \implies s \sqsubseteq f(x) \forall x \in U$. \\\\

 Per \textbf{a} e \textbf{b} $\implies \forall x \in U $ $f(x) \in U$. Ovvero $U$ è chiuso rispetto ad $f$.


Siccome $f: U \to U $ e U è un reticolo completo in quanto intervallo di un reticolo completo ($L$), e per \textbf{Punto 1} e \textbf{Punto 2} vale che $f:U \to U$ con $U$ reticolo completo ha lub e glb  in U $\implies f$ ha lub e glb in $U$ e lub=greatest fixed point e glb=least fixed point di $f$ in U.

%to do dimostra che least fixed point è il lub di S , che vale lo stesso per glb e che vale per ogni sottoinsieme di P => p è complete lattica

Siccome $s \sqsubseteq x \forall x \in U$ allora s è il greatest lower bound di U, quindi s = least fixed point di $f$ in $U$. Allora $S$ ha lub in $F$ perchè il lub di $S = s =$ least fixed point di $f$ in $U$ e quindi $\in F$.\\

Analogamente i può dimostrare che $S$ ha glb in $F$, e siccome questo vale $\forall S \subseteq P \implies P$ è un reticolo completo
\end{comment}

\subsection{Knaster Tarski in CCS}

Sia $R \subseteq Proc \times Proc$ tc se $(P, Q) \in R$ allora:
\begin{itemize}
    \item $\forall P \passo P' \exists Q \passo Q'$ tc $(P', Q') \in R$
    \item $\forall Q \passo Q' \exists P \passo P'$ tc $(P', Q') \in R$
\end{itemize}

Definisco $$F(R) = \{ (P,Q) | \text{proprietà sopra} \} $$

Allora se $(P,Q) \in R \implies (P,Q) \in F(R)$, ovvero $R \subseteq F(R)$ cioè $R$ è bisismulazione.

F è monotona perchè più grande è R più coppie ci sono in F(R) (mantiene la relazione di $\sqsubseteq$ da $R$ a $F$).

$R$ è bisimulazione sse $R \subseteq F(R)$. $ \sim= \cup$bisimulazioni cioè $\sim = \cup \{ R |R $ è bisimulazione$\} = \cup \{R | R \subseteq F(R)\} $ quindi $\sim = $ max Fixed Point$(F)$.
%TODO scrivi meglio e dì perché è reticolo completo

\end{document}
